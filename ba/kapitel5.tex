\chapter{Referenzen und Zitate}\label{cha-ref}

Im Prinzip kann in \LaTeX auf alles referenziert werden was ein Label hat. Dies kann ein Kapitel oder Abschnitt sein, siehe Kapitel \ref{cha-ref} und Anhang \ref{app-A}, eine Formel wie die von Bernoulli (\ref{eqn-bernoulli}), eine Graphik wie Abbildung \ref{fig-buch}, eine Tabelle wie Tabelle \ref{tab-mathe} oder sogar Punkte einer Aufz"ahlung, vgl.~\ref{enum-ebene}.

Noch eleganter sind Zitate. Man zitiert am besten auf ein K"urzel welches sich aus den ersten Buchstaben des Erstautors und der Jahreszahl zusammensetzt wie \cite{sch04}. Die Seitenzahl kann als Option angegeben werden \cite[S.~30]{sch04}. Verwenden Sie BibTeX, so erscheinen nur die verwendeten Literaturstellen im Literaturverzeichnis und "uberdies k"ummert sich dann \LaTeX um die richtige Reihenfolge und Formatierung der Quellen - egal ob Buch \cite{sch04}, Artikel \cite{kat06} oder Dokumentation \cite{meh10}.
