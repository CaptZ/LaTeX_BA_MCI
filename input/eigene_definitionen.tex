% Listings zum einbinden von Programmcode
\usepackage{listings}

% Zus"atzliche Tabelleneigenschaften
\usepackage{tabularx}
\newcolumntype{L}[1]{>{\raggedright\arraybackslash}p{#1}} % linksb"undig mit Breitenangabe
\newcolumntype{C}[1]{>{\centering\arraybackslash}p{#1}} % zentriert mit Breitenangabe
\newcolumntype{R}[1]{>{\raggedleft\arraybackslash}p{#1}} % rechtsbündig mit Breitenangabe
\newcolumntype{Z}[1]{>{\centering\arraybackslash}m{#1}} % horizontal und vertikal zentriert
\newcolumntype{Y}[1]{>{\raggedright\arraybackslash}m{#1}} % linksb"undig und vertikal zentriert

% Tabellenzeilen verbinden
\usepackage{multirow}

% Tabellenh"ohe
\renewcommand{\arraystretch}{1.2}

% Mechanik-Bibliotheken einbinden
\usepackage{input/elementlibrary}
\usepackage{input/3dstructuralanalysis}

% Konstruktionszeichnungen-Bibliothek einbinden
\usepackage{input/konstruktionszeichnungen_mit_tikz}

% Schaltungszeichnungen-Bibliothek einbinden
\usepackage{circuitikz}

% Textblöcke Auskommentieren
\usepackage{verbatim}

% Anhänge und pdf Dokument anhängen
\usepackage{appendix}
\usepackage{pdfpages} 

%en­able a LATEX source file to gen­er­ate ex­ter­nal files
\usepackage{filecontents}

% Standardformatierung f"ur Diagramme
\usepackage{pgfplots}
\pgfplotsset{/pgf/number format/use comma} % Komma als Dezimaltrennzeichen
\pgfkeys{/pgf/number format/.cd, set thousands separator={\,}} % Tausendertrennzeichen
\pgfplotsset{
every axis/.append style={
thick,
grid style={gray!50,thin},
tick style={black,thick},
legend style={font=\footnotesize},
compat=newest,
},
every axis plot post/.append style={black,mark options={solid,draw=black,fill=white}},
every x tick label/.style={yshift=-1pt},	% X-Achsenbeschriftung ein wenig nach unten verschieben
}

% siunitx für SI-Einheiten (Befehle \SI \si \num ...)
\usepackage{siunitx}
\sisetup{
output-decimal-marker={,},
per-mode=reciprocal,
exponent-product=\cdot,
retain-explicit-plus,
range-phrase = {\dots},
separate-uncertainty,
list-separator={; },
list-final-separator={; }
}

\newcommand{\dezt}{,}		% Dezimaltrennzeichen
\newcommand{\atlbe}{}		% Ende der Abbildungs-/Tabellen-/Listingsbenennung
\newcommand{\multz}{\cdot}	% Multiplikationszeichen
\newcommand{\taut}{\,}		% Tausendertrennzeichen

\newcommand{\bildbreite}{\textwidth}		% Bildbreite
\newcommand{\bildhoehe}{\textheight/3}		% Bildhoehe
\newcommand{\diagbreite}{\textwidth}		% Diagrammbreite
\newcommand{\diaghoehe}{\diagbreite/1.618}	% Diagrammhoehe (Goldener Schnitt)

% auxillary symbols
\renewcommand{\tilde}{\symbol{126}}
\newcommand{\define}{\stackrel{!}{=}}
\renewcommand{\equiv}{\,\widehat{=}\,}
\newcommand{\subsubsubsection}{\textbf}
\newcommand{\re}{\mathrm{Re}}
\newcommand{\pr}{\mathrm{Pr}}
\newcommand{\st}{\mathrm{St}}
\newcommand{\fr}{\mathrm{Fr}}
\newcommand{\nus}{\mathrm{Nu}}
\newcommand{\gr}{\mathrm{Gr}}
\newcommand{\ra}{\mathrm{Ra}}
%\renewcommand{\not}{\not}
\newcommand{\im}{i}
\newcommand{\ariwam}{ARiWaM}
\newcommand{\matlab}{MATLAB}

% mathematical operators
\newcommand{\grad}{\,\mathrm{grad}\,}
\renewcommand{\div}{\,\mathrm{div}\,}
\newcommand{\rot}{\,\mathrm{rot}\,}
\newcommand{\lap}{\Delta}
\newcommand{\laplace}[1]{\mathscr{L}\left\{#1\right\}}
\newcommand{\trans}{^T}
\newcommand{\norm}{\psarc[linewidth=0.5pt](0,0){0.4}{0}{90}\psdot[dotsize=0.1](0.15,0.15)}