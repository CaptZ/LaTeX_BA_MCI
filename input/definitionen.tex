\ifthenelse{\boolean{english}}
{}{
	% deutsche Anpassungen
	\usepackage[T1]{fontenc} % Aktiviert EC-Schriftarten
	\usepackage{ngerman} % Deutsche Einstellungen
	\usepackage[utf8]{inputenc}
}
\usepackage{url}

%\usepackage[dvips]{graphicx}
\usepackage[tmargin=1in,bmargin=1in,lmargin=1.25in,rmargin=1.25in]{geometry}
%\usepackage{titlesec} %nicht Kompatibel mit der Dokumentenklasse
\usepackage{xcolor}
\usepackage[overload]{textcase}
\definecolor{MSBlue}{rgb}{.204,.353,.541}
\definecolor{MSLightBlue}{rgb}{.31,.506,.741}
\usepackage{booktabs}

% Set formats for each heading level
%\titleformat*{\section}{\rmfamily\bfseries\huge\color{MSBlue}\lowercase}
%\titleformat{\section}[hang]{\rmfamily\bfseries\huge\color{MSBlue}\lowercase}{\thesection}{1em}{}[]
%\titleformat{\subsection}{\large\bfseries\sffamily\uppercase}{\thesubsection}{1em}{}
%\titleformat{\subsubsection}{\sffamily\bfseries}{\thesubsubsection}{1em}{}

\usepackage{graphicx}
\usepackage{psfrag,tikz} % epsfig,
\usetikzlibrary{arrows,calc}
\usepackage{pst-all} % malen
\usepackage{color} % farben
\usepackage{exscale,relsize}
\usepackage{fancyhdr}
\usepackage[small]{caption}
\usepackage{scrhack}
\usepackage{float}
\usepackage{amsmath}
\usepackage{units}
\usepackage{subfigure}
\usepackage{wallpaper}
\usepackage{pst-all}
\usepackage{rotating}
\usepackage{amsmath,amssymb,amsfonts,amstext}
\usepackage{mathrsfs}
\usepackage{pgfplots}
	\pgfplotsset{compat=1.14}
\usepackage{makeidx}
\usepackage[bookmarks=true,bookmarksnumbered=true]{hyperref}
\usepackage{colortbl}	%% Farbe für Tabellen
\usepackage{listings} 	%% Listing für Programmiersprachen
\lstset{
  language={C},
  %basicstyle=\tiny  %% kleiner Sourcecode
  %commentstyle=\tiny  %% nur Kommentar klein --> sieht scheiße aus
}

\newcommand{\babel}[2]{\ifthenelse{\boolean{english}}{#1}{#2}}
\newcommand{\mif}{\quad\mathrm{\babel{if}{falls}}\quad}
\newcommand{\with}{\quad\mathrm{\babel{with}{mit}}\quad}
\newcommand{\for}{\quad\mathrm{\babel{for}{f"ur}}\quad}

% color set
\definecolorseries{foo}{rgb}{last}[rgb]{1.0,0.0,0.0}[rgb]{0.0,0.0,1.0}
\resetcolorseries[16]{foo}

% format specifications
\renewcommand{\emph}{\textit}
\newcommand{\file}{\textit}
\newcommand{\cmd}{\texttt}
\newcommand{\ten}{\boldsymbol}
%\newcommand{\unit}{\mathrm}
\newcommand{\lemma}{\textit}
\newcommand{\deutsch}[1]{german: \textit{#1}}
\renewcommand{\index}{\emph}

% Command path to graphic files
\newcommand{\gpath}{./grafics}
\newcommand{\bsppath}{../uebungen/beispiele}

%\renewcommand{\labelenumi}{\alph{enumi})}

\setlength{\parindent}{0em}
\setlength{\parskip}{1.5ex plus0.5ex minus0.5ex}
\setlength{\captionmargin}{3em}

% counters
\newcounter{example}
\newcommand{\exampletext}{Beispiel }
\newcommand{\example}[1]{\underline{\exampletext \arabic{chapter}.\arabic{example}:} #1 \addtocounter{example}{1}}
\newcounter{exercise}
\newcommand{\exercisetext}{Aufgabe }
\newcommand{\exercise}[1]{\underline{\exercisetext \arabic{chapter}.\arabic{exercise}:} #1 \stepcounter{exercise}}
\newcommand{\cchapter}[1]{\chapter{#1} \setcounter{example}{1} \setcounter{exercise}{1}}
\newcommand{\solution}{\textit{L\"osung:} }
%\newcounter{beispiel}
%\setcounter{beispiel}{1}		% Nummer des ersten Beispiels
\newboolean{student}
\newcounter{enumcount}
\newcommand{\resume}[1]{\begin{#1} \setcounter{enumi}{\value{enumcount}}}
\newcommand{\pause}[1]{\setcounter{enumcount}{\value{enumi}} \end{#1}}

% Kopfzeilen frei gestaltbar
\usepackage{fancyhdr}
\lfoot[\fancyplain{}{}]{\fancyplain{}{}}
\rfoot[\fancyplain{}{}]{\fancyplain{}{}}
\cfoot[\fancyplain{}{\footnotesize\thepage}]{\fancyplain{}{\footnotesize\thepage}}
\lhead[\fancyplain{}{\footnotesize\nouppercase\leftmark}]{\fancyplain{}{}}
\chead{}
%\rhead[\fancyplain{}{}]{\fancyplain{}{\footnotesize\nouppercase\sc\leftmark}} %\sc commando is obsolete
\rhead[\fancyplain{}{}]{\fancyplain{}{\footnotesize\nouppercase\leftmark}}

% Farben im Dokument m"oglich
\usepackage{color}

% Schriftart Helvetica
\usepackage{helvet}
\renewcommand{\familydefault}{cmss}

% anderdhalbfacher Zeilenabstand
\usepackage{setspace}
\onehalfspacing

% Graphiken einbinden: hier für pdflatex
%\usepackage[dvips]{graphicx}

% verbesserte Floating Plazierung
\usepackage{float}

% Überprüfung des Layouts
\usepackage{layout}

\usepackage{array}

% Höhe und Breite des Textkörpers etwas grösser definieren
\usepackage[tmargin=1in,bmargin=1in,lmargin=1.25in,rmargin=1.25in]{geometry}

% Einrückung von und Abstand zwischen Absätzen
\setlength{\parindent}{0em}
\setlength{\parskip}{1.5ex plus0.5ex minus0.5ex}

% weniger Warnungen wegen überfüllter Boxen
\tolerance = 9999
\sloppy

% Counter für die Nummerierung
\newcounter{romancount}