% Nomenklatur folgendermassen aufgebaut:
%
% - Symbolverzeichnis
%	- Deutsche Symbole: erzeugt "uber \nmD["Sortiervariable"]{"Symbol"}{"Bezeichnung"}{"Einheit"}
%	- Griechische Symbole: erzeugt "uber \nmG["Sortiervariable"]{"Symbol"}{"Bezeichnung"}{"Einheit"}
%	- Index: erzeugt "uber \nmI["Sortiervariable"]{"Index"}{"Bezeichnung"}
% - Seitenumbruch
% - Abk"urzungsverzeichnis: erzeugt "uber \nmI["Sortiervariable"]{"Abk"urzung"}{"Bezeichnung"}
%
%
% Automatisierte Nomenklatur
% In Texmaker als eigenen Befehl erzeugen: makeindex.exe %.nlo -s nomencl.ist -o %.nls
% Anschliessend XeLaTeX, danach MakeIndex und dann wieder XeLaTex durchlaufen lassen
\usepackage[intoc]{nomencl}\makenomenclature
% Zeilenabstände verkleinern
\setlength{\nomitemsep}{-\parsep}
%
% Variablen ("Uberschriften):
\ifthenelse{\boolean{english}}{
	\renewcommand{\nomname}{Nomenclature}
}{
	\renewcommand{\nomname}{Nomenklatur}
}
\newcommand\symbV{\ifthenelse{\boolean{english}}{List of symbols}{Symbolverzeichnis}}
\newcommand\abkV{\ifthenelse{\boolean{english}}{List of abbreviations}{Abk"urzungsverzeichnis}}
\newcommand\IndV{\ifthenelse{\boolean{Indices}}{List of symbols}{Indizes}}
\newcommand\symboll{\ifthenelse{\boolean{english}}{Symbol}{Symbol}}
\newcommand\bezeichnung{\ifthenelse{\boolean{english}}{Designation}{Bezeichnung}}
\newcommand\einheit{\ifthenelse{\boolean{english}}{Unit}{Einheit}}
\newcommand\indexx{\ifthenelse{\boolean{english}}{Index}{Index}}

\renewcommand*{\nompreamble}{\markboth{\nomname}{\nomname}}
\newcommand*{\chUnit}[1]{\hfill\makebox[1.7cm]{#1\hfill}}
\newcommand*{\VarWidth}[1]{\parbox[c]{2cm}{#1}}
\newcommand*{\DescrWidth}[1]{\parbox[c]{7cm}{#1}}
\newcommand*{\chSubtitle}[1]{\item[\large\bfseries#1]}
\newcommand*{\chnomSeq}[1]{\csname chSymb#1\endcsname}
\renewcommand*{\nomgroup}{\chnomSeq}

% Symbolverzeichnis
% A = deutsche Symbole, B = griechische Symbole
\newcommand*{\chSymbA}{\addsec*{\symbV}%
	\item[\VarWidth{\textbf{\symboll}}]%
	\textbf{\bezeichnung}\chUnit{\textbf{\einheit}}}
\newcommand*{\chSymbB}{\vspace{0.2cm}}
% Indizes
\newcommand*{\chSymbC}{\vspace{0.5cm}%
	\item[\VarWidth{\textbf{\indexx}}]%
	\textbf{\bezeichnung}}

% Abk"urzungsverzeichnis
\newcommand*{\chSymbD}{\addsec*{\abkV}}
	
% Formatierung f"ur Eintr"age mit Einheit
\newcommand*{\chwithUnit}[4]{\nomenclature[#1]{\VarWidth{#2}}{\DescrWidth{#3}\chUnit{#4}}}
% Formatierung f"ur Eintr"age ohne Einheit
\newcommand*{\chwithoutUnit}[3]{\nomenclature[#1]{\VarWidth{#2}}{#3}}

\newcommand*{\nmD}[4][]{\chwithUnit{A#1}{#2}{#3}{#4}} % Deutsche Symbole
\newcommand*{\nmG}[4][]{\chwithUnit{B#1}{#2}{#3}{#4}} % Griechische Symbole
\newcommand*{\nmI}[3][]{\chwithoutUnit{C#1}{#2}{#3}} % Indizes
\newcommand*{\nmA}[3][]{\chwithoutUnit{D#1}{#2}{#3}} % Abk"urzungsverzeichnis