%============================================================
% 3D-Ansichten
%============================================================
\tikzstyle{isometric}=[x={(0.710cm,-0.410cm)},y={(0cm,0.820cm)},z={(-0.710cm,-0.410cm)}]
\tikzstyle{dimetric} =[x={(0.935cm,-0.118cm)},y={(0cm,0.943cm)},z={(-0.354cm,-0.312cm)}]
\tikzstyle{dimetric2}=[x={(0.935cm,-0.118cm)},z={(0cm,0.943cm)},y={(+0.354cm,+0.312cm)}]
\tikzstyle{trimetric}=[x={(0.926cm,-0.207cm)},y={(0cm,0.837cm)},z={(-0.378cm,-0.507cm)}]

%============================================================
% FLUSSDIAGRAMM
%============================================================
% BL"OCKE
\tikzstyle{block} = [rectangle,thick,draw,fill=white,text width=10em, text centered, rounded corners, minimum height=2.5em]
\tikzstyle{invblock} = [rectangle,text width=10em, text centered, rounded corners, minimum height=2em]
\tikzstyle{sysblock} = [rectangle,thick,draw,fill=white,text width=8em, text centered, minimum height=5em]

% PFEILE
\tikzstyle{arrow} = [thick,->,>=latex]
\tikzstyle{d_arrow} = [thick,<->,>=latex]

%============================================================
% MECHANISCHE SYSTEME
%============================================================
% KRAFT
\tikzstyle{force>}=[line width = 1mm,->,>=latex]
\tikzstyle{force<}=[line width = 1mm,<-,>=latex]

% FEDER
\tikzstyle{spring}=[thick,decorate,decoration={zigzag,pre length=1cm,post length=1cm,segment length=8,amplitude=10pt}]
\tikzstyle{spring_short}=[thick,decorate,decoration={zigzag,pre length=0.5cm,post length=0.5cm,segment length=8,amplitude=10pt}]

\tikzstyle{spring_small}=[thick,decorate,decoration={zigzag,pre length=0.5cm,post length=0.5cm,segment length=8,amplitude=10pt}]

% D"AMPFER
\tikzstyle{damper}=[thick,decoration={markings,  
  mark connection node=dmp,
  mark=at position 0.46 with 
  {
    \node (dmp) [thick,inner sep=0pt,transform shape,rotate=-90,minimum width=20pt,minimum height=15pt,draw=none] {};
    \draw [thick] ($(dmp.north east)+(10pt,0)$) -- (dmp.south east) -- (dmp.south west) -- ($(dmp.north west)+(10pt,0)$);
    \draw [thick] ($(dmp.north)+(0,-8pt)$) -- ($(dmp.north)+(0,8pt)$);
  }
}, decorate]

% GROUND
\tikzstyle{ground}=[fill,pattern=north east lines,draw=none,minimum width=0.75cm,minimum height=0.3cm]

% MASSE
\tikzstyle{mass} = [rectangle,thick,draw,minimum width=3.6cm,minimum height=2cm,anchor=south]

%============================================================
% SCHRAFFUR
%============================================================
\tikzset{schraffiert45/.style={pattern=north east lines,pattern color=black}} % 45° Schraffur

\tikzset{schraffiert135/.style={pattern=north west lines,pattern color=black}} % 135° Schraffur

% BENUTZERDEFINIERTE SCHRAFFUR
\newlength{\schraffurabstand}
\newlength{\schraffurdicke}
% BENUTZERDEFINIERTE SCHRAFFUR 135°
% defining the new dimensions
% declaring the keys in tikz
\tikzset{schraffurabstand/.code={\setlength{\schraffurabstand}{#1}},
         schraffurdicke/.code={\setlength{\schraffurdicke}{#1}}}
% setting the default values
\tikzset{schraffurabstand=3mm,
         schraffurdicke=0.25mm}
% declaring the pattern
\pgfdeclarepatternformonly[\schraffurabstand,\schraffurdicke]% variables
   {bschraffiert135}% name
   {\pgfqpoint{-2\schraffurdicke}{-2\schraffurdicke}}% lower left corner
   {\pgfqpoint{\dimexpr\schraffurabstand+2\schraffurdicke}{\dimexpr\schraffurabstand+2\schraffurdicke}}% upper right corner
   {\pgfqpoint{\schraffurabstand}{\schraffurabstand}}% tile size
   {% shape description
    \pgfsetlinewidth{\schraffurdicke}
    \pgfpathmoveto{\pgfqpoint{0pt}{\schraffurabstand}}
    \pgfpathlineto{\pgfqpoint{\dimexpr\schraffurabstand+0.15pt}{-0.15pt}}
    \pgfusepath{stroke}
   }

%============================================================
% MCI-Seitige Symbole
%============================================================
% electrical networks
\newcommand{\con}{\pscircle[linewidth=2pt,fillstyle=solid,fillcolor=black](0,0){0.1}}
\newcommand{\pin}{\pscircle[linewidth=2pt,fillstyle=solid](0,0){0.15}}
\newcommand{\link}{\psline[linewidth=2pt]}
\newcommand{\linkd}{\psline[linestyle=dashed]}
\newcommand{\arcd}{\psarc[linestyle=dashed]}
\newcommand{\arrow}{\psline[linewidth=2pt,arrowsize=8pt]{->}}
\newcommand{\cs}{\pscircle[linewidth=2pt](0,0){0.5}\arrow(0,-0.4)(0,0.4)\link(0,-0.5)(0,-2)\link(0,0.5)(0,2)}
\newcommand{\vs}{\pscircle[linewidth=2pt](0,0){0.5}\psline(-0.2,-0.1)(0.2,-0.1)\psline(-0.2,0)(0.2,0)\pscurve(-0.2,0.1)(-0.1,0.15)(0,0.1)(0.1,0.05)(0.2,0.1)\link(0,-0.5)(0,-2)\link(0,0.5)(0,2)}
\newcommand{\res}{\psframe[linewidth=2pt](-0.3,-1)(0.3,1)\link(0,-1)(0,-2)\link(0,1)(0,2)}
\newcommand{\ind}{\psframe[linewidth=2pt,fillstyle=solid,fillcolor=black](-0.3,-1)(0.3,1)\link(0,-1)(0,-2)\link(0,1)(0,2)}
\renewcommand{\cap}{\link(-0.5,-0.1)(0.5,-0.1)\link(-0.5,0.1)(0.5,0.1)\link(0,-0.1)(0,-2)\link(0,0.1)(0,2)}
\newcommand{\opv}{\pspolygon[linewidth=2pt](-1,-1.3)(-1,1.3)(1.3,0)\link(-1,-0.8)(-2,-0.8)\link(-1,0.8)(-2,0.8)\link(1.3,0)(2,0)\rput(-0.7,0.8){$-$}\rput(-0.7,-0.8){$+$}}

% mechanical networks
\newcommand{\mass}{\psframe[linewidth=2pt](-0.5,-0.5)(0.5,0.5)\link(0,0.5)(0,1)\link(0,-0.5)(0,-1)}
\newcommand{\spring}{\link(0,-1)(0,-0.8)(-0.3,-0.7)(0.3,-0.5)(-0.3,-0.3)(0.3,-0.1)(-0.3,0.1)(0.3,0.3)(-0.3,0.5)(0.3,0.7)(0,0.8)(0,1)}
\newcommand{\damp}{\link(-0.5,0.4)(-0.5,-0.3)(0.5,-0.3)(0.5,0.4)\link(-0.42,0.2)(0.42,0.2)\link(0,0.2)(0,1)\link(0,-0.3)(0,-1)}
\newcommand{\hatch}{\psframe[fillstyle=vlines,linestyle=none,hatchwidth=0.25pt]}
\newcommand{\clamp}{\hatch(-0.5,-0.3)(0.5,0)\link(-0.5,0)(0.5,0)}
\newcommand{\fixed}{\hatch(-0.5,-0.7)(0.5,-0.4)\link(-0.5,-0.4)(0.5,-0.4)\link(-0.25,-0.4)(0,0)(0.25,-0.4)\pscircle[linewidth=2pt,fillstyle=solid](0,0){0.1}}
\newcommand{\loose}{\hatch(-0.5,-0.7)(0.5,-0.5)\link(-0.5,-0.4)(0.5,-0.4)\link(-0.5,-0.5)(0.5,-0.5)\link(-0.25,-0.4)(0,0)(0.25,-0.4)\pscircle[linewidth=2pt,fillstyle=solid](0,0){0.1}}
\newcommand{\axial}{\link(-0.3,-0.3)(-0.3,-0.1)(0.3,-0.1)(0.3,-0.3)\link(-0.3,0.3)(-0.3,0.1)(0.3,0.1)(0.3,0.3)}
\newcommand{\force}{\psline[linewidth=2pt,arrowsize=6pt]{->}}
\newcommand{\rmom}{\psarc[arrowsize=6pt,linewidth=2pt]{<-}}
\newcommand{\lmom}{\psarc[arrowsize=6pt,linewidth=2pt]{->}}
\newcommand{\stress}{\psline[arrowsize=6pt]{->}}
\newcommand{\bbeam}{\psframe[fillstyle=solid,fillcolor=lightgray]}
\newcommand{\rod}{\psline[linewidth=3pt]}
\newcommand{\disp}{\psline[linewidth=2pt,arrowsize=6pt]{->}}	% displacement

% hydraulical networks
\newcommand{\pump}{\pscircle[linewidth=2pt,fillstyle=solid,fillcolor=white](0,0){1}\pspolygon[fillstyle=solid,fillcolor=black](0,1)(-0.15,0.3)(0.15,0.3)}
\newcommand{\motor}{\pscircle[linewidth=2pt,fillstyle=solid,fillcolor=white](0,0){0.8}\rput(0,0){M}}
\newcommand{\throttle}{\psarc[linewidth=2pt](-3,0){2.7}{-15}{15}\psarc[linewidth=2pt](3,0){2.7}{165}{195}}
\newcommand{\limiter}{\psframe[linewidth=2pt,fillstyle=solid,fillcolor=white](-0.7,-0.7)(0.7,0.7)\psline{->}(-0.5,-0.4)(-0.5,0.1)(0.5,-0.25)}

% diagrams (block and s-plane)
\newcommand{\axis}{\psline[arrowsize=6pt,arrowinset=0]{->}}
\newcommand{\measure}{\psline[linewidth=0.5pt,arrowsize=4pt]}
\newcommand{\xtick}[1]{\measure(#1,-0.1)(#1,0.1)}
\newcommand{\ytick}[1]{\measure(-0.1,#1)(0.1,#1)}
\newcommand{\aang}{\psarc[linewidth=0.5pt,arrowsize=4pt]{<->}}
\newcommand{\tf}[1]{\psframe[linewidth=2pt](-1,-0.75)(1,0.75)\rput(0,0){#1}}
\newcommand{\tff}[2]{\psframe[fillstyle=solid,fillcolor=lightgray](-1,0)(1,0.75)\psframe[linewidth=2pt](-1,-0.75)(1,0.75)\rput(0,0.35){#1}\rput(0,-0.35){#2}}
\newcommand{\sat}{\psframe[linewidth=2pt](-1,-0.75)(1,0.75)\link(-0.8,-0.5)(-0.5,-0.5)(0.5,0.5)(0.8,0.5)}
\newcommand{\add}[4]{\pscircle[linewidth=2pt,fillstyle=solid,fillcolor=white](0,0){0.5}\rput(-0.5,-0.75){#1}\rput(-0.75,0.5){#2}\rput(0.5,0.75){#3}\rput(0.75,-0.5){#4}}
\newcommand{\pole}{\link(-0.15,-0.15)(0.15,0.15)\link(-0.15,0.15)(0.15,-0.15)}
\newcommand{\zero}{\pscircle[linewidth=2pt,fillstyle=solid](0,0){0.15}}
\newcommand{\textblock}[1]{\psframebox[linewidth=2pt]{\parbox[c][2cm][c]{3cm}{\centering #1}}}

% thermodynamic systems
\newcommand{\cv}{\psframe[linestyle=dashed]}
\newcommand{\sys}{\psframe[fillcolor=white,fillstyle=solid,linewidth=2pt]}
\newcommand{\iso}{\psframe[linewidth=2pt,fillstyle=vlines]}
\newcommand{\candle}{\psframe[linewidth=0.5pt](-0.15,-0.5)(0.15,0.3)\pscurve[linewidth=0.5pt](0,0.8)(-0.1,0.5)(0,0.3)(0.1,0.5)(0,0.8)\pscurve[linewidth=0.5pt](0,0.6)(-0.08,0.4)(0,0.3)(0.08,0.4)(0,0.6)}
\newcommand{\flux}{\psline[arrowsize=10pt,linewidth=4pt]{->}}
\newcommand{\ppoint}[1]{\pscircle(0,0){0.3}\rput[c](0,0){#1}}

% aerodynamics
\newcommand{\source}{\pin\rput{0}(0,0){\arrow(0.5,0)(1,0)}\rput{45}(0,0){\arrow(0.5,0)(1,0)}\rput{90}(0,0){\arrow(0.5,0)(1,0)}\rput{135}(0,0){\arrow(0.5,0)(1,0)}\rput{180}(0,0){\arrow(0.5,0)(1,0)}\rput{225}(0,0){\arrow(0.5,0)(1,0)}\rput{270}(0,0){\arrow(0.5,0)(1,0)}\rput{315}(0,0){\arrow(0.5,0)(1,0)}}

% CFD
\newcommand{\cell}[1]{\definecolor{cellc}{gray}{#1}\psframe[linewidth=2pt,fillstyle=solid,fillcolor=cellc](0,0)(1,1)}