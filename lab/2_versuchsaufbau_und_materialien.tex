\ifthenelse{\boolean{english}}{
	\chapter{Test Preparation \& Materials}
}{
	\chapter{Versuchsaufbau \& Materialien}
}
\label{cha:versuchsaufbau}

In \cite{meh10} sind viele interessante Informationen enthalten.

Es kann auch auf \cite{laser2016} verwiesen werden.

In \figurename~\ref{fig:diagramm_beispiel} ist eine Beispielkurve aufgetragen.


\begin{figure}[H]
\centering
\begin{tikzpicture}

\begin{axis}[
%scaled x ticks={base 10:3},
%scaled y ticks={base 10:-3},
width=0.9\diagbreite,
height=1.0\diaghoehe,
%
%title = \textbf{Geschwindigkeitsverlauf gemessen an den Messpunkten am Druckbalken},
%
xlabel=$x\,/\,\si{m}$,
ylabel={$f(x)\,/\,\si{m/s}$},
%yticklabel pos=right,
%
xmax = 6,
xmin = -6,
ymax = 50,
ymin = -10,
xtick={-6,-5,...,6},
%ytick={0,-1,-2,-3,-4,-5,-6,-7,-8,-9,-10},
legend style = {
at = {(0.02,0.97)},
anchor=north west,
cells={anchor=west},
},
%
grid = both,
%minor tick num=0,
minor x tick num={1},
minor y tick num={1},
]
	\addplot[only marks,mark=o,samples=20]{x^2 - x +4};
		\addlegendentry{$f(x) = x^2 - x +4$};
\end{axis}
\end{tikzpicture}
\caption[Kurze Bezeichnung des Diagramms f"ur das Abbildungsverzeichnis]{Beispiel eines Diagramms\atlbe}
\label{fig:diagramm_beispiel}
\end{figure}


In \Gl~\ref{eqn:gleichung_beispiel} ist das dynamische Grundgesetz beschrieben. Die \Gl~\ref{eqn:integral1} zeigt den Allgemeine Zusammenhang von Ort, Geschwindigkeit und Beschleunigung. 

\begin{equation}
	F = m \cdot a
\label{eqn:gleichung_beispiel}
\end{equation}
% Eintr"age in das Symbolverzeichnis
%\nmD[F]{$F$}{Kraft}{\si{N}}	% Bereits im Symbolverzeichnis
\nmD[m]{$m$}{Masse}{\si{kg}}
\nmD[a]{$a$}{Beschleunigung}{\si{m/s^2}}

\begin{equation}
	z(t) = \int\limits_0^t v(t)dt = \iint\limits_0^t a(t)dt
\label{eqn:integral1}
\end{equation}
% Eintr"age in das Symbolverzeichnis
\nmD[v]{$v$}{Geschwindigkeit}{\si{m/s}} 
\nmD[t]{$t$}{Zeit}{\si{s}}
\nmD[z]{$z$}{Auslenkung}{\si{m}}

Der Code, um \figurename~\ref{fig:diagramm_beispiel2} zu erstellen soll zeigen, wie externe Daten z.~B. Messdaten in ein Diagramm eingef"ugt werden k"onnen. Das geht sehr komfortabel.


\begin{figure}[H]
\centering
\begin{tikzpicture}
\begin{axis}[
%scaled x ticks={base 10:3},
%scaled y ticks={base 10:-3},
width=0.9\diagbreite,
height=1.0\diaghoehe,
%
%title = \textbf{Geschwindigkeitsverlauf gemessen an den Messpunkten am Druckbalken},
%
xlabel=$x\,/\,\si{s}$,
ylabel={$Messdaten~und~f(x)\,/\,\si{m}$},
%yticklabel pos=right,
%
xmax = 2*pi+0.001,
xmin = 0,
%ymax = 50,
%ymin = -10,
%xtick={-6,-5,...,6},
%ytick={0,-1,-2,-3,-4,-5,-6,-7,-8,-9,-10},
legend style = {
at = {(0.98,0.97)},
anchor=north east,
cells={anchor=west},
},
%
grid = both,
%minor tick num=0,
minor x tick num={1},
minor y tick num={1},
]
% Diagrammdaten
\pgfplotstableread{diagrammdaten/messdaten_beispiel.txt}\DatatableMessdatenBeispiel
	
	\addplot[only marks,mark=o] table[y = y_m] from \DatatableMessdatenBeispiel ;
		\addlegendentry{Messdaten};
		
	\addplot[samples=500,domain=0:2*pi]{sin(deg(x))};
		\addlegendentry{Fitfunktion $f(x) = \mathrm{sin}(x)$};
\end{axis}
\end{tikzpicture}
\caption[Kurze Bezeichnung des zweiten Diagramms f"ur das Abbildungsverzeichnis]{Zweites Beispiel eines Diagramms\atlbe}
\label{fig:diagramm_beispiel2}
\end{figure}


\begin{equation}
\dot{\textbf{x}} =
\begin{bmatrix}
  \dot{a} \\
  \dot{b} \\
  \ddot{c} \\
  \ddot{d} \\
  \end{bmatrix}
  =
\begin{bmatrix} 
  1 & 0 & 0 & 0 \\
  0 & 1 & 0 & 0 \\
  0 & 0 & \frac{1}{2} & \frac{1}{3} \\
  0 & 0 & 0 & 0
  \end{bmatrix}
\begin{bmatrix}
  x \\
  x_{ab} \\
  \dot{x} \\
  \dot{x}_{ab} \\
  \end{bmatrix}
 +
\begin{bmatrix}
  0 \\
  0 \\
  0 \\
  1 \\
  \end{bmatrix}
 \ddot{z}_{ab}
\label{eqn:matrix_beispiel_1}
\end{equation}


\begin{equation}    
\left( \begin{array}{rrr|r}
     1 & \nicefrac{4}{3} & \nicefrac{5}{3} & \nicefrac{22}{3}\\
     0 & -\nicefrac{8}{3} & \nicefrac{2}{3} & -\nicefrac{14}{3}\\
     0 & \nicefrac{2}{3} & \nicefrac{4}{3} & \nicefrac{8}{3}\\
\end{array}\right)
\label{eqn:matrix_beispiel_2}
\end{equation}


Wenn eine Matrix \textbf{A} ben"otigt wird kann sie so aussehen.

%Abschnitt Zentrierung Anfang
{\par\centering
%\begin{equation} %wird nicht verwendet da der Platz nicht ausreicht
$A$ =
$  \begin{bmatrix}
    0 & 0 & 0 & 1 & 0 & 0 & 0 & 0 \\
	0 & 0 & 0 & 0 & 1 & 0 & 0 & 0 \\
	0 & 0 & 0 & 0 & 0 & 1 & 0 & 0 \\
	0 & 0 & 0 & 0 & 0 & 0 & 1 & 0 \\
	0 & 0 & 0 & 0 & 0 & 0 & 0 & 1 \\
	0 & 0 & 0 & \frac{a}{b} & \frac{aa}{b} & 0 & 0 & 0 \\
	\frac{aaa}{b} & \frac{aaaa}{b} & 0 & \frac{aaaaa}{b} & \frac{aaaaaaa}{b} & \frac{abbccccc}{b} & \frac{adddddddd}{b} & 0 \\
	\frac{a}{b} & 0 & 0 & 0 & \frac{a}{b} & \frac{a}{b} & 0 & 0 \\
	\frac{a}{b} & \frac{a}{b} & 0 & \frac{a}{b} & 0 & \frac{a}{b} & \frac{a}{b} \\
	0 & 0 & 0 & 0 & 0 & 0 & 0 & 0
  \end{bmatrix}$
%\label{eqn_hol:state_1} %wird nicht verwendet da der Platz nicht ausreicht
%\end{equation} 		 %wird nicht verwendet da der Platz nicht ausreicht
%Abschnitt Zentrierung Ende
\par}

%Abstand kann händich eingegeben, sollte aber vermieden werden
\vspace*{0.5cm}

\begin{equation}
b = \begin{bmatrix} 0 & 0 & 0 & 0 & 0 & 0 & 0 & 0 & 0 & 1
  \end{bmatrix}^T
\label{eqn:matrix_beispiel_3}
\end{equation}

Matrix \Gl~\ref{eqn:matrix_beispiel_4} ist als small gekennzeichnt und \Gl~\ref{eqn:matrix_beispiel_5} hat die Normale gr"o{\ss}e.

\begin{equation}
  \mbox{\small $\begin{pmatrix}
 	0	&1	&4	&0\\
 	1	&0	&0	&4\\
 	3	&4	&0	&1\\
 	1	&1	&0	&3
  \end{pmatrix}$}
\label{eqn:matrix_beispiel_4}
\end{equation}

\begin{equation}
  \mbox{ $\begin{pmatrix}
 	0	&1	&4	&0\\
 	1	&0	&0	&4\\
 	3	&4	&0	&1\\
 	1	&1	&0	&3
  \end{pmatrix}$}
\label{eqn:matrix_beispiel_5}
\end{equation}
